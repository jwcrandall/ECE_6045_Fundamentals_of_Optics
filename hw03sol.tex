\documentclass[main.tex]{subfiles}
\begin{document}

\textbf{Exercise 7.13}\\ %5th7.16*
\textbf{Q\&A} A standing wave is given by $E=100\sin(\frac{2}{3}\pi x) \cos( 5\pi t )$. Determine two waves that can be superimposed to generate it.\\

We state a trigonometric product identity in Equation \ref{eq:713_1}

\begin{equation}\label{eq:713_1}
\sin(\alpha)\cos(\beta) = \frac{\sin(\alpha + \beta) + \sin(\alpha - \beta)}{2}
\end{equation}

and rewrite the $\sin$ and $\cos$ component of the given standing wave in the form of Equation \ref{eq:713_1} in Equation \ref{eq:713_2}.

\begin{equation}\label{eq:713_2}
\sin(\frac{2}{3}\pi x)\cos(5\pi t) = \frac{\sin(\frac{2}{3}\pi x + 5\pi t) + \sin(\frac{2}{3}\pi x - 5\pi t)}{2}
\end{equation}

We can now define $E$ as the superposition of two waves in Equation \ref{eq:713_2} where $50 \sin(\frac{2}{3}\pi x + 5\pi t)$ is traversed in the negative x axis and $50 \sin(\frac{2}{3}\pi x - 5\pi t)$ is traversed in the positive x axis.

\begin{equation}\label{eq:713_2}
E = 50 \sin(\frac{2}{3}\pi x + 5\pi t) + 50 \sin(\frac{2}{3}\pi x - 5\pi t)
\end{equation}

\textbf{Exercise 6.3}\\
\textbf{Q\&A} Write an  expression for the thickness $d$ of a double-convex lens such that its focal length is infinite.\\

We begin by defining the ray transfer matrix for a thick lens in Equation \ref{eq:63_1}.

\begin{equation}\label{eq:63_1}
\begin{bmatrix}
    \alpha_2 n_{air}\\
    x_2
\end{bmatrix}
=
\begin{bmatrix}
    1   &   -\frac{1-n}{R_2} \\
    0   &   1
\end{bmatrix}
\begin{bmatrix}
    1   &   0 \\
    \frac{d}{n}   &   1
\end{bmatrix}
\begin{bmatrix}
    1   &   -\frac{n-1}{R_1} \\
    0   &   1
\end{bmatrix}
\begin{bmatrix}
    \alpha_{1}n_{air} \\
    x_1
\end{bmatrix}
\end{equation}

We then multiply matrices from Equation \ref{eq:63_1} and arrive at Equation \ref{eq:63_2}

\begin{equation}\label{eq:63_2}
\begin{bmatrix}
    \alpha_2 n_{air}\\
    x_2
\end{bmatrix}
=
\begin{bmatrix}
    1  - \frac{d}{n}\frac{1-n}{R_2}   &   -\frac{1-n}{R_2} \\
    \frac{d}{n}   &   1
\end{bmatrix}
\begin{bmatrix}
    1   &   -\frac{n-1}{R_1} \\
    0   &   1
\end{bmatrix}
\begin{bmatrix}
    \alpha_{1}n_{air} \\
    x_1
\end{bmatrix}
\end{equation}

and multiply matrices from Equation \ref{eq:63_2} and arrive at Equation \ref{eq:63_3}.

\begin{equation}\label{eq:63_3}
\begin{bmatrix}
    \alpha_2 n_{air}\\
    x_2
\end{bmatrix}
=
\begin{bmatrix}
    1  - \frac{d}{n}\frac{1-n}{R_2}   &  -\frac{n-1}{R_1}  + \frac{d}{n}\frac{1-n}{R_2}\frac{n-1}{R_1} -\frac{1-n}{R_2}   \\
    \frac{d}{n}                       &   1  - \frac{d}{n}\frac{n-1}{R_1}
\end{bmatrix}
\begin{bmatrix}
    \alpha_{1}n_{air} \\
    x_1
\end{bmatrix}
\end{equation}

where Equation \ref{eq:63_3} can be simplified to Equation \ref{eq:63_4}.

\begin{equation}\label{eq:63_4}
\begin{bmatrix}
    \alpha_2 n_{air}\\
    x_2
\end{bmatrix}
=
\begin{bmatrix}
    1  - \frac{d}{n}\frac{1-n}{R_2}   &  -\left[ (n-1) \left(\frac{1}{R_1}-\frac{1}{R_2}\right)  + \frac{d}{n}\frac{(n-1)^2}{R_1 R_2} \right]\\
    \frac{d}{n}                       &   1  - \frac{d}{n}\frac{n-1}{R_1}
\end{bmatrix}
\begin{bmatrix}
    \alpha_{1}n_{air} \\
    x_1
\end{bmatrix}
\end{equation}

From Equation \ref{eq:63_4} we can define the Effective focal Length in Equation \ref{eq:63_5}

\begin{equation}\label{eq:63_5}
-\frac{1}{EFL} = -\left[ (n-1) \left(\frac{1}{R_1}-\frac{1}{R_2}\right)  + \frac{d}{n}\frac{(n-1)^2}{R_1 R_2} \right]
\end{equation}

and given that the focal length is infinite we can state that $\frac{1}{\infty} = 0$ and rewrite \ref{eq:63_5} in the form of Equation \ref{eq:63_6}

\begin{equation}\label{eq:63_6}
-\left(\frac{1}{R_1}-\frac{1}{R_2}\right) =  \frac{d}{n}\frac{(n-1)^2}{R_1 R_2} \right]
\end{equation}

solve for $d$ in Equation \ref{eq:63_7}

\begin{equation}\label{eq:63_7}
\frac{(R_1 - R_2) }{R_1 R_2} \frac{R_1 R_2}{(n-1)^2}} n =  d
\end{equation}

and simplify in in Equation \ref{eq:63_8} to arrive at distance $d$.

\begin{equation}\label{eq:63_8}
d = \frac{n (R_1 - R_2)} {(n-1)^2} 
\end{equation}

%%% NOTE the square in the denominator may be incorrect.

\textbf{Exercise 7.10}\\
\textbf{Q\&A} The electric field of a standing electromagentic plane wave is given by $E(x,t) = 2E_0 \sin (kx) \cos(\omega t)$. Derive an expression for $B(x,t)$. (You might want to take another look at Section 3.2). Make a sketch of the standing wave.\\



\textbf{Exercise 6.17}\\ %6.23 in 5thed
\textbf{Q\&A} Show that the planar surface of a concave-planar or convex-planar lens doesn't contribute to the system matrix.\\

We begin by defining the ray transfer matrix for a thick convex planar lens in Equation \ref{eq:617_1}.

\begin{equation}\label{eq:617_1}
\begin{bmatrix}
    \alpha_2 n_{air}\\
    x_2
\end{bmatrix}
=
\begin{bmatrix}
    1   &   -\frac{1-n}{R_2} \\
    0   &   1
\end{bmatrix}
\begin{bmatrix}
    1   &   0 \\
    \frac{d}{n}   &   1
\end{bmatrix}
\begin{bmatrix}
    1   &   -\frac{n-1}{R_1} \\
    0   &   1
\end{bmatrix}
\begin{bmatrix}
    \alpha_{1}n_{air} \\
    x_1
\end{bmatrix}
\end{equation}

We know that for a plane surface $R_2 = \infty$ which contributes to a unit matrix in Equation \ref{eq:617_2} and proves that that the planar surface does not contribute to the system matrix. 

\begin{equation}\label{eq:617_2}
\begin{bmatrix}
    \alpha_2 n_{air}\\
    x_2
\end{bmatrix}
=
\begin{bmatrix}
    1   &   -\frac{1-n}{R_2} \\
    0   &   1
\end{bmatrix}
\begin{bmatrix}
    1   &   0 \\
    \frac{d}{n}   &   1
\end{bmatrix}
\begin{bmatrix}
    1   &   0 \\
    0   &   1
\end{bmatrix}
\begin{bmatrix}
    \alpha_{1}n_{air} \\
    x_1
\end{bmatrix}
\end{equation}\\

\textbf{Exercise 6.18}\\ %6.24 in 5th ed
\textbf{Q\&A} Compute the system matrix for a thick biconvex lens of index $1.5$ having radii of $0.5$ and $0.25$ and a thickness of $0.3$ (in any units you like). Check that $|A| = 1$.\\

We begin by defining the system matrix for a thick biconvex lens in Equation \ref{eq:617_1}.

\begin{equation}\label{eq:617_1}
A =
\begin{bmatrix}
    1   &   -\frac{1-n}{R_2} \\
    0   &   1
\end{bmatrix}
\begin{bmatrix}
    1   &   0 \\
    \frac{d}{n}   &   1
\end{bmatrix}
\begin{bmatrix}
    1   &   -\frac{n-1}{R_1} \\
    0   &   1
\end{bmatrix}
\end{equation}


which by matrix multiplication we arrive at Equation \ref{eq:617_2}

\begin{equation}\label{eq:617_2}
A=
\begin{bmatrix}
    1  - \frac{d}{n}\frac{1-n}{R_2}   &  -\left[ (n-1) \left(\frac{1}{R_1}-\frac{1}{R_2}\right)  + \frac{d}{n}\frac{(n-1)^2}{R_1 R_2} \right]\\
    \frac{d}{n}                       &   1  - \frac{d}{n}\frac{n-1}{R_1}
\end{bmatrix}
\end{equation}

and by substituting known values we populate Equation \ref{eq:617_3}

\begin{equation}\label{eq:617_3}
A=
\begin{bmatrix}
    1  - \frac{0.3}{1.5}\frac{1-1.5}{0.25} &  -\left[ (1.5-1) \left(\frac{1}{0.5}-\frac{1}{0.25}\right)  + \frac{0.3}{1.5}\frac{(1.5-1)^2}{(0.5)(0.25)} \right]\\
    \frac{0.3}{1.5}                       &   1  - \frac{0.3}{1.5}\frac{1.5-1}{0.5}
\end{bmatrix}
\end{equation}

which equates to Equation \ref{eq:617_4}.

\begin{equation}\label{eq:617_4}
A=
\begin{bmatrix}
    1.4 &  0.6\\
    0.2 &   0.8
\end{bmatrix}
\end{equation}

By taking the determinate of Equation \ref{eq:617_4} we are left with Equation \ref{eq:617_5}.

\begin{equation}\label{eq:617_5}
|A|= (1.4)(0.8) - (0.6)(0.2) = 1.12 - 0.12 = 1
\end{equation}

\textbf{Complex arithmetic}\\
\textbf{Q\&A} In Wave Optics, the use of complex numbers, in particular phasors, is prevalent because it considerably simplifies calculations of interference and diffraction. The goal of this exercise is to remind you of some basic complex arithmetic. Let $z_1 = 3 + i4$, $z_2 = 1-i$, $z_3 = 5e^{i\pi /3}$, and $z_4=5e^{i4\pi/3}$. Compute, in the easiest way possible, and without the use of electronic calculators, the following quantities:

\begin{itemize}
  \item $|z_1|$, $|z_2|$, $|z_3|$, $|z_4|$, $\angle z_1$, $\angle z_2$, $\angle z_3$, $\angle z_4$;
  \item $\angle -z_1$, $\angle z_2^*$,$\angle -z_3$, $\angle z_4^*$; ($z^*$ denotes the complex conjugate of the complex number $z$;)
  \item $z_1 + z_2$, $z_1^* + z_2$, $z_3 + z_4$, $z_1 - z_4^*$;
  \item the magnitude and phases of  $z_1 z_2$, $z_3 z_4$, $z_3 / z_4$, $\sqrt{z_3}$;
  \item $z_1 + e^{i \pi}$, $e^{i \pi / 2}z_2$, $e^{i\pi}z_3$, $\sqrt{e^{-i\pi}z_4}$.
\end{itemize}

\textbf{Wave superposition}\\
\textbf{Q\&A} Consider the following two waves,

\begin{equation}\label{eq:ws_1}
f_1(x,z,t) = 5\cos \left(\frac{2 \pi}{17} \left[ z+\frac{x^2}{2z} \right] -2\pi 10t \right)
\end{equation}

\begin{equation}\label{eq:ws_2}
f_2(x,z,t) = 5\cos \left(\frac{2 \pi}{17} \left[ z+\frac{(x-5)^2}{2z} \right] -2\pi 10t + \frac{\pi}{3} \right)
\end{equation}

What is the physical interpretation of these waves? Be as detailed as possible.\\

If these waves approximately satisfy the (Helmholtz) Wave Equation, what is the phase velocity?\\

Using trigonometric identities, calculate the wave that results from the "coherent superposition" of the two waves, $i.e. f(x,z,t) \equiv f_1(x,z,t) + f_2(x,z,t)$.\\

Now express each wave as a phasor, add the two phasors and compare the result to the phasor of $f(x,z,t)$ from part (b).\\

\textbf{Plane waves and phasor representations}\\
\textbf{Q\&A} Throughout this problem, by "real expression" of a wave we mean the space-time representation, e.g. $f(x,y,z,t) = A\cos(kz-\omeega t)$ is a plane wave of wave-vector magnitude $k$ and frequency $\omega$ propagating in the direction of the $\hat{z}$ coordinate axis. By "phasor expression" we mean the complex representation of the wave, e.g. $Ae^{ikz}$ for the same wave.\\

\textbf{Q\&A} Write the real and phasor expression for a plane wave $f_1(x,y,z,t)$ propagating at an angle $\ang{30}$ relative to the $hat{z}$ axis on the $xz$-plane ($i.e.$, the plane $y=0$). The wavelength is $\lambda = \SI{1}{\mu \meter}$, and the wave speed is $c = \SI{3e8}{\metre \second^{-1}}$.\\

\textbf{Q\&A} Write the real and phasor expressions for a place wave $f_2(x,y,z,t)$ of the same wavelength and wave speed as $f_1$ but propagating at angle $\ang{60}$ relative to the $\hat{z}$ axis on the $yz$-plane.

\end{document}