\documentclass[main.tex]{subfiles}
\begin{document}

\textbf{Exercise 7.13}\\
\textbf{Q\&A} A standing wave is given by $E=100\sin(\frac{2}{3}\pi x) \cos( 5\pi t)$. Determine two waves that can be superimposed to generate it.\\

\textbf{Exercise 6.3}\\
\textbf{Q\&A} Write an  expression for the thickness $d$ of a double-convex lens such that its focal length is infinite.\\

\textbf{Exercise 7.10}\\
\textbf{Q\&A} The electric field of a standing electromagentic plane wave is given by $E(x,t) = 2E_0 \sin (kx) \cos(\omega t)$. Derive an expression for $B(x,t)$. (You might want to take another look at Section 3.2). Make a sketch of the standing wave.\\

\textbf{Exercise 6.17}\\
\textbf{Q\&A} Show that the planar surface of a concave-planar or convex-planar lens doesn't contribute to the system matrix.\\

\textbf{Exercise 6.18}\\
\textbf{Q\&A} Compute the system matrix for a thick biconvex lens of index $1.5$ having radii of $0.5$ and $0.25$ and a thickness of $0.3$ (in any units you like). Check that $|A| = 1$.\\

\textbf{Complex arithmetic}\\
\textbf{Q\&A} In Wave Optics, the use of complex numbers, in particular phasors, is prevalent because it considerably simplifies calculations of interference and diffraction. The goal of this exercise is to remind you of some basic complex arithmetic. Let $z_1 = 3 + i4$, $z_2 = 1-i$, $z_3 = 5e^{i\pi /3}$, and $z_4=5e^{i4\pi/3}$. Compute, in the easiest way possible, and without the use of electronic calculators, the following quantities:

\begin{itemize}
  \item $|z_1|$, $|z_2|$, $|z_3|$, $|z_4|$, $|z_2|$,
  \item Another entry in the list
\end{itemize}

\end{document}