\documentclass[main.tex]{subfiles}
\begin{document}

\begin{enumerate}

% -----------------------------------------------------
% First problem
% -----------------------------------------------------

\item \textbf{\underline{Fermat's Principles}} requires: (More than 1 option could be valid) (\textit{3 points}) \\
\bigcirc \text{ Path length of light is the minimum}\\
\bigcirc \text{ Rays travel in a straight line in a uniform media}\\
\bigcirc \text{ Rays can travel in curved trajectory}\\
\bigcirc \text{ The trajectory of a ray has to be piece-wise differentiable}\\
\bigcirc \text{ Path trajectory of the beam can be non-differentiable}\\
\bigcirc \text{ The trajectory of the beam has to be continuous}\\

% -----------------------------------------------------
% Second problem
% -----------------------------------------------------

\item \textbf{\underline{Perfect focusing}}: Which catoptric (mirror) system can perfectly focus a ray bundle of rays parallel to its axis? Demonstrate and discuss what are the main inconveniences (drawbacks) of such a system and limitations. Can it be used for magnifying an off-axis object? Please draw a schematic which represents it.  \textit{[Additional point: Derive the ray transfer matrix for such an optical object when a ray bundle parallel to the axis is impinging on it] (4+1 points)} \\

% -----------------------------------------------------
% Third problem
% -----------------------------------------------------

\item In regards with \textbf{\underline{paraxial assumption}} which of the following statements are correct: (More than 1 option could be valid) (\textit{3 points}) \\
\bigcirc \text{ All the lenses are considered thin lenses}\\
\bigcirc \text{ Small angle approximated is valid because $z>>x$, $z>>y$ and very large R}\\
\bigcirc \text{ Curves are approximation lenses are not affected by chromatic aberration}\\
\bigcirc \text{ Ray transfer matrix can be applied only if paraxial assumption holds}\\
\bigcirc  \text{Paraxial assumption can be just used in geometrical optics approximation } \\ \text{\quad but not in wave optics.}

% -----------------------------------------------------
% Four problem
% -----------------------------------------------------

% -----------------------------------------------------
% Five problem
% -----------------------------------------------------

% -----------------------------------------------------
% Six problem
% -----------------------------------------------------

% -----------------------------------------------------
% Seven problem
% -----------------------------------------------------

% -----------------------------------------------------
% Eight problem
% -----------------------------------------------------

% -----------------------------------------------------
% Nine problem
% -----------------------------------------------------

% -----------------------------------------------------
% Ten problem
% -----------------------------------------------------

\end{document}